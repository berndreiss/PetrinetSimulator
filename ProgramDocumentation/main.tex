\documentclass[10pt, a4paper]{article}

\usepackage[english]{babel}
\usepackage{csquotes}
%\usepackage{hologo}

%\usepackage{float}

%\usepackage{babelbib}

%\usepackage{amsmath}
%\usepackage{amssymb}
%\usepackage{amsthm}

\usepackage{graphicx}
\graphicspath{ {./Pictures/} }

\begin{document}


\section{Introduction to the Program}
%\label{sec1}

\subsection{The Menu}

\subsubsection{File}

The \textit{File} menu consists of the following entries:

\begin{itemize}

\item \textbf{New:} opens a new empty file for editing. 
\item \textbf{Open:} opens saved file from directory.
\item \textbf{Open in new Tab:} same as open but in new tab.
\item \textbf{Reload:} reload the currently opened file.
\item \textbf{Save:} save changes.
\item \textbf{Save as:} save changes and choose directory/name.
\item \textbf{Analyse++:} analyse many petrinets at once. Results are printed in the text area.
\item \textbf{Close:} close currently opened file.
\item \textbf{Exit:} close the program.

The \textit{Edit} menu consists of the following entries:

\end{itemize}

\begin{itemize}
\item \textbf{Open Editor:} opens the Editor.
\item \textbf{Close Editor:} closes the Editor and reverts back to Viewer.
\item \textbf{Change Look and Feel:} change between Nimbus and Metal look and feel.
\end{itemize}

The \textit{Help} menu only consists of the element \textit{Info}, which shows information about the Java version used and the user directory.


\subsection{The Viewer}

\subsubsection{Petrinet Toolbar}

\includegraphics[width=5cm]{Viewer_Petrinet_Toolbar.png}


\subsubsection{Reachbility Graph Toolbar}

\includegraphics[width=5cm]{Viewer_Reachability_Toolbar.png}

%\hologo{LaTeX} 
%\ref{sec1}

\subsection{The Editor}

\subsubsection{Petrinet Toolbar}

\includegraphics[width=5cm]{Editor_Toolbar.png}

%\begin{itemize}

%\item{}
%\end{itemize}


%\begin{table}[t]

%\centering


%\begin{tabular}{| c | c | c | c |}
%\hline
%		& Hobbit		& Ringträger		& Gemeinschaft des Ringes	\\ \hline\hline

%\end{tabular}

%\caption{Verschiedene Charaktere und Gruppierungen in \textit{Der Herr der Ringe}.}
%\label{tab:LOTR}

%\end{table}


%\cite{jamesetal13}

 
%\bibliographystyle{babplain-fl}

%\bibliography{Literatur}

\end{document}

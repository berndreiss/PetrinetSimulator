\documentclass[10pt, a4paper]{article}

\usepackage[a4paper, margin=2cm]{geometry}
\usepackage[english]{babel}
\usepackage{csquotes}
\usepackage{hyperref}
\usepackage[dvipsnames]{xcolor}
\usepackage{ulem}
\usepackage{setspace}

%\usepackage{hologo}

%\usepackage{float}

%\usepackage{babelbib}

%\usepackage{amsmath}
%\usepackage{amssymb}
%\usepackage{amsthm}

\usepackage{graphicx}
\graphicspath{ {./Pictures/} }

\begin{document}

\pagenumbering{roman} 
\textcolor{white}{ }

\vspace{-1cm}

%\begin{figure}[h]
%	\flushright 
%	\includegraphics[width=5cm]{Bilder/logop}
%\end{figure}

\vspace{1.5cm}
\begin{center}

\sffamily

\fontsize{12pt}{1}{\upshape{}}
	\vspace{2cm}

\fontsize{20pt}{1}{\textbf{PETRINET SIMULATOR}}
	\vspace{0.3cm}

\fontsize{18pt}{1}{\upshape{}}
	\vspace{2.5cm}

\fontsize{14pt}{1}{\upshape{}}
	\vspace{0.2cm}

\fontsize{12pt}{1}{\textbf{Program Documentation}}
	\vspace{0.2cm}

\fontsize{12pt}{1}{\upshape{}}
	\vspace{2.5cm}

\fontsize{18pt}{1}{\upshape{}}
	\vspace{4.5cm}

\fontsize{16pt} {1}{\upshape{}}
	\vspace{1cm}

\fontsize{16pt} {1}{\uline{\textbf{}}}
	\vspace{1cm}

\fontsize{12pt} {1}{\upshape{}}


\end{center}

\thispagestyle{empty}
\newpage
%\afterpage{\blankpage}



 
%\normalem 
% \blankpage

%\textcolor{white}{彩蛋}
\thispagestyle{empty}
\newpage 

\tableofcontents \thispagestyle{plain} \newpage


\pagenumbering{arabic} \rmfamily \onehalfspace 

\section{Introduction to the Program}
%\label{sec1}

The Petrinet Simulator is able to display a petrinet and gradually build a reachability graph by firing transitions. The reachability graph will show whether it encountered a state that is unbounded and mark the beginning and ending nodes on the path signifying the unboundedness of the petrinet. Additionally there is the option to analyse a given petrinet with regards to its boundedness and build the reachability graph with the click of a button. Furthermore it provides an user interface for editing petrinets. 

The following sections give an overview of the menus and toolbars and the functionality provided by it. 

\subsection{The Menu}

\subsubsection{File}

The \textit{File} menu consists of the following entries:

\begin{itemize}

\item \textbf{New:} opens a new empty file for editing -> automatically opens the Editor view (see \ref{label:editor}). 
\item \textbf{Open:} opens saved file from directory.
\item \textbf{Open in new Tab:} same as open but in a new tab.
\item \textbf{Reload:} reload the currently opened file.
\item \textbf{Save:} save changes.
\item \textbf{Save as:} save changes and choose directory/name.
\item \textbf{Analyse++:} analyse many petrinets at once. Results are printed in the text area.
\item \textbf{Close:} close currently opened file.
\item \textbf{Exit:} close the program.

\end{itemize}


\subsubsection{Edit}


The \textit{Edit} menu consists of the following entries:


\begin{itemize}
\item \textbf{Open Editor:} opens the Editor.
\item \textbf{Close Editor:} closes the Editor and reverts back to Viewer.
\item \textbf{Change Look and Feel:} change between Nimbus and Metal look and feel.
\end{itemize}

\subsubsection{Help}

The \textit{Help} menu only consists of the element \textit{Info}, which shows information about the Java version used and the user directory.


\subsection{The Viewer}

\subsubsection{Petrinet Toolbar}

\includegraphics[width=5cm]{Viewer_Petrinet_Toolbar.png}


\subsubsection{Reachbility Graph Toolbar}

\includegraphics[width=5cm]{Viewer_Reachability_Toolbar.png}

%\hologo{LaTeX} 
%\ref{sec1}

\subsection{The Editor}
\label{label:editor}
\subsubsection{Petrinet Toolbar}

\includegraphics[width=5cm]{Editor_Toolbar.png}

%\begin{itemize}

%\item{}
%\end{itemize}


\section{Exploring the Technical Depths}


\subsection{General Design of the Program}

\subsection{Packages}



%\begin{table}[t]

%\centering


%\begin{tabular}{| c | c | c | c |}
%\hline
%		& Hobbit		& Ringträger		& Gemeinschaft des Ringes	\\ \hline\hline

%\end{tabular}

%\caption{Verschiedene Charaktere und Gruppierungen in \textit{Der Herr der Ringe}.}
%\label{tab:LOTR}

%\end{table}


%\cite{jamesetal13}

 
%\bibliographystyle{babplain-fl}

%\bibliography{Literatur}

\end{document}
